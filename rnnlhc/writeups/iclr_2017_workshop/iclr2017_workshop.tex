\documentclass{article} % For LaTeX2e
\usepackage{iclr2017_workshop,times}
\usepackage{hyperref}
\usepackage{url}


\title{Exploring the role of Deep Learning for High Energy Physics}


\author{Mayur Mudigonda  \\
Redwood Center for Theoretical Neuroscience\\
UC Berkeley\\
\texttt{{mudigonda}@.edu} \\
\And
Steve Farell\\
\texttt{{sfarell}@lbl.gov} \\
\And
Prabhat\\
Lawrence Berkeley National Lab \\
Berkeley\\
\texttt{prabhat@lbl.gov}
\And
Paolo Calafiura\\
Lawrence Berkeley National Lab \\
Berkeley \\
\texttt{calafiura@lbl.gov}
}

% The \author macro works with any number of authors. There are two commands
% used to separate the names and addresses of multiple authors: \And and \AND.
%
% Using \And between authors leaves it to \LaTeX{} to determine where to break
% the lines. Using \AND forces a linebreak at that point. So, if \LaTeX{}
% puts 3 of 4 authors names on the first line, and the last on the second
% line, try using \AND instead of \And before the third author name.

\newcommand{\fix}{\marginpar{FIX}}
\newcommand{\new}{\marginpar{NEW}}

\begin{document}


\maketitle

\begin{abstract}
The abstract paragraph should be indented 1/2~inch (3~picas) on both left and
right-hand margins. Use 10~point type, with a vertical spacing of 11~points.
The word \textsc{Abstract} must be centered, in small caps, and in point size 12. Two
line spaces precede the abstract. The abstract must be limited to one
paragraph.
\end{abstract}

\begin{document}


\maketitle
Deep Learning has played a phenomenal role in making advances in many fields suc as computer vision, speech recognition, robotics. In this work, we explore the role of Deep Learning for problems in High Energy Physics. First, we present the complexities of the problem in detecting particles. Next, we present preliminary results on the applications of LSTMs to tracking particles in a detector array. We hope, with this work, to reach out to the broader machine learning community to both present our findings and seek out methods for solving challenging problems in high energy physics (HEP)

\section{The pattern recognition problems in High Energy Physics Detectors}
Detectors typically are subterranean and are arranged in concentric layers around a core. Typically, an atoms of ___ are bombarded with a neutrons. The resulting splitting of the atom produces various subatomic particles. These particles exit the detector with different momenta, charge and directions. The momenta and charge together offer insight into the nature of the particles. 

One of the pattern recognition tasks involved is to explain the trajectories of all particles from a single experiment. That is, given a 3D image $I(x,y,z)$, triplet of inputs, where each pixel has a binary value with 1 signifying a hit on the detector layer

The similarities in problems between those explored in computer vision, robotics and the HEP field are obvious. The obvious differences lie in the fact that in the case of HEP-LHC typically we would need to estimate the parameters of millions of tracks in parallel. Further, the required reliability of a model is significantly higher. For example, the existing state of the art methods can detect tracks with a reliability greater than 99\%. 

One advantage of solving this problem is that it could potentially have applications in real time vision and robotics. For example, to articulate a very high DOF actuator such as an elephant trunk, a humanoid arm, inferences of this nature are required to be solved. 

\section{Modeling}

\section{Results}

\bibliography{iclr2017_workshop}
\bibliographystyle{iclr2017_workshop}

\end{document}
